\documentclass[a4paper,11pt]{article}
\usepackage[authoryear]{natbib}

% define the title
\author{T. Kooij}
\title{Simulatie van fotondetectie in HiSPARC}
\begin{document}
% generates the title
\maketitle
% insert the table of contents
\tableofcontents

\section{Samenvatting}
Ja, het kan!

\section{Inleiding}
Geladen leptonen (elektronen en muonen) worden in HiSPARC stations efficient gedetecteerd met vinyltolueen scintilatorplaten. Doordat geladen deeltjes zeer veel interacties met materie hebben verliezen ze veel energie in de dectector. De detectie kans is 1 en de energieafgifte per deeltje is reproduceerbaar.

Detectie van fotonen is moeizaam. De werkzame doorsnedes van de interacties zijn klein, zodat de kans op interactie slechts enkele procentpunt is \citep*{Pennink:2010}. De energieafgifte per interactie kan echter groot zijn. Daarom is de energieafgifte per foton variabel.

In een pulsehoogtehistogram van een HiSPARC detector station zijn bijdragen van fotonen te identificeren \citep*{Pennink:2010}. In de simulatie van EAS op HiSPARC detectoren met behulp van CORSIKA en sapphire wordt geen rekening gehouden met de bijdragen van fotonen. Fotonen zijn wel aanwezig in de CORSIKA output, maar worden niet meegenomen in sapphire. Daarom is de interactie van fotonen met de HiSPARC detectoren onderzocht om de detectorreponse op fotonen in the bouwen in de sapphire simulaties.

\section{Theorie}
In de HiSPARC scintilator platen word fotonen (gamma's) gedetecteerd met een energie tussen 100 keV en 10 MeV \citep*{Steijger2010-gammas}. Er zijn drie mechanismen waarmee dergelijke gamma's energieverliezen in een scintilatorplaat: Het foto-elektrisch effect, compton verstrooiing en paarvorming.

Het






\bibliography{mybib}{}
\bibliographystyle{plain}


\end{document}
